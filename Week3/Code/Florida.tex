\documentclass[12pt]{article} % Use 10pt, 9pt, or \documentclass[8pt]{article} % Use 10pt, 9pt, or 8pt for smaller text
\usepackage[utf8]{inputenc}
\usepackage{geometry}
\geometry{
  a4paper,
  total={170mm,257mm},
  left=10mm,
  right=10mm,
  top=5mm, % Adjusted top margin
}
\usepackage{graphicx}
\usepackage{titling}

% Title customization
\setlength{\droptitle}{-50pt} % Reduce space above the title
\title{Feeling the Burn: Is Florida Getting Warmer?}

\usepackage{fancyhdr}
\fancypagestyle{plain}{% Default header/footer clearing
    \fancyhf{} % clear all header and footer fields
}
\makeatletter
\def\@maketitle{%
  \newpage
  \null
  \vskip 0.5em % Reduced space between the top and title
  \begin{center}%
    \let \footnote \thanks
    {\LARGE \@title \par}%
    \vskip 0.5em % Reduced space below the title
  \end{center}%
  \par
  \vskip 0.5em}
\makeatother

\usepackage{lipsum}
\usepackage{cmbright}

\begin{document}

\maketitle

\section*{Results} 

I accessed data on Key West Annual Mean Temperature from the TMQB GitHub on the 3rd November 2024. The data had 100 observations spanning from the year 1901-2000. The mean temperature was 25.31 degrees (SD 0.5, range 23.75-26.35). I found a positive, statistically significant correlation between year and temperature (Figure 1). This result was confirmed with the application of a non-parametric permutation test which validated the observed temperature trend.  

\textbf{Pearson Correlation}\\
A correlation was found between temperature and year ($t = 6.2$, $df = 98$, $p$-value $= 1.123 \times 10^{-8}$). The correlation coefficient was 0.5331. 

\begin{figure}[h]
    \centering
    \includegraphics[width=0.5\textwidth]{Rplot.pdf}
    \caption{The correlation between temperature and the years 1900-2000. }
    \label{fig:sample_image}
\end{figure}

\textbf{Permutation Analysis}\\
I performed 1000 permutations shuffling temperature data. No random correlation permutations had a higher correlation coefficient than 0.5331. I calculated the $p$-value by dividing the number of results higher than the observed correlation. The p-value was 0. The mean correlation coefficient of the null distribution was close to 0 (mean = -0.008, SD = 0.10), indicating no correlation when temperature data were randomized. Therefore, the likelihood of obtaining a correlation of 0.5331 or higher by chance alone is extremely low. This suggests that the correlation between temperature and year is likely not a result of a random pattern.

\end{document}

