CMEECourseWork

Title: Week 6/7 GIS

Brief Description:

This repository contains coursework for the Computational Methods in Ecology and Evolution (CMEE), specifically focusing on biological computing tasks using Linux and shell scripting. The exercises are based on TheMulQuaBio course notes (https://mhasoba.github.io/TheMulQuaBio/intro.html) from the Biological Computing course at the Department of Life Sciences, Imperial College London.

Project Structure and Usage: The repository contains code scripts located in the Code folder. The Data folder includes input files used by some scripts, while the sandbox folder contains experimental files and is not essential to the main coursework. Output files are generated in the results folder for this week’s coursework.

Project Structure

    Code folder: Contains all scripts.
    Data folder: Includes input files used by scripts.
    Results folder: Stores output files generated by the scripts.
    Sandbox folder: Used for experimental work (not essential for the coursework).

Languages: 
R version 4.4.0 

Dependencies: 
terra
sf
units
geodata
openxlsx

Installation:

To clone this repository, use the following command:
bash
git clone git@github.com:Saskiapearce/CMEECourseWork.git

Project Structure and Usage:
The repository contains primary scripts located in the Code folder. The Data folder includes input files used by some scripts, while the sandbox folder contains experimental files and is not essential to the main coursework. Output files are generated in the results folder for this week’s coursework.

GIS Practical 1.R

    Purpose: This script provides a comprehensive framework for Geographic Information System 	     (GIS) analysis, including vector and raster data processing, visualization, and spatial analysis. The workflow integrates multiple R packages to manipulate spatial data, create custom geometries, project coordinates, and perform rasterization. Additionally, it incorporates climate data visualization and topographical mapping.
    Usage: Run the script directly in R.
    Input: Shapefiles in ../Data/uk/ for saving/loading UK and Ireland vector geometries.
    Climate data (WHOSIS_000001.csv) and raster datasets (etopo_25.tif) in ../Data/.
    Output:
Plots

    Custom polygons and points for UK and Ireland regions.
    Raster maps for temperature, elevation, and aggregated/disaggregated values.
    Temperature maps for January, July, and annual maximum temperatures.

Files

    Shapefiles for vector data:
        ../Data/uk/uk_eire_WGS84.shp
        ../Data/uk/uk_eire_BNG.shp
    Raster files in GeoTIFF and ASCII formats:
        ../Data/uk/uk_raster_BNG_interp.tif
        ../Data/uk/uk_raster_BNG_ngb.asc

Example Run:

    r

    source("GIS Practical 1.R")
    # prints: polyons, and points for the UK and Ireland regions, Raster maps for temperature and elevation and aggregated values 
    # 